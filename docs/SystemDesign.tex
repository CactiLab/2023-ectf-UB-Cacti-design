%==============================================================================
\documentclass[11pt,oneside,onecolumn,letterpaper]{article}
\usepackage{times}
\usepackage[paperwidth=8.5in, paperheight=11in,
top=2.5cm, bottom=2.6cm, left=2.58cm, right=2.53cm]{geometry}
%\setlength{\textheight} {9.00in}
%\setlength{\textwidth}  {6.40in}
%\setlength{\topmargin}  {-0.50in}
%%\setlength{\headheight} {0.00in}
%%\setlength{\headsep}     {0.40in}
%\setlength{\oddsidemargin}{-0.010in}
%\setlength{\evensidemargin}{-0.00in}
%==============================================================================
%\usepackage{algorithm}
\usepackage{amssymb}
\usepackage{color,soul}
\usepackage{booktabs}
\usepackage{graphicx}
\usepackage{latexsym}
\usepackage{subfigure}
\usepackage{wrapfig}
\usepackage{amsmath}
\usepackage{amsthm}
\usepackage[hyphens]{url}
\usepackage{pifont}
\usepackage{xcolor}
\usepackage{colortbl}
\usepackage{indentfirst}
\usepackage[lined, boxed, linesnumbered]{algorithm2e}
\usepackage[square, comma, sort&compress, numbers]{natbib}

\newcounter{alg}
\newenvironment{enum-ref}{
\begin{list}%
{[\arabic{alg}]} {\usecounter{alg}
  \setlength{\leftmargin} {0.25in}
  \setlength{\labelwidth} {0.30in}
  \setlength{\rightmargin}{0.00in}
  \setlength{\topsep}     {0.00in}}
}{\end{list}}

\newenvironment{enum-number}{
\begin{list}%
{\arabic{alg})} {\usecounter{alg}
  \setlength{\leftmargin} {0.25in}
  \setlength{\labelwidth} {0.30in}
  \setlength{\rightmargin}{0.00in}
  \setlength{\topsep}     {0.00in}}
}{\end{list}}

\newenvironment{enum-nonum}{
\begin{list}%
{$\bullet$} {
  \setlength{\leftmargin} {0.25in}
  \setlength{\labelwidth} {0.30in}
  \setlength{\rightmargin}{0.00in}
  \setlength{\topsep}     {0.00in}}
}{\end{list}}

\newcommand{\ziming}[1]{%
  \begingroup
  \definecolor{hlcolor}{RGB}{20, 255, 20}\sethlcolor{hlcolor}%
  \textcolor{black}{\hl{\textit{\textbf{Ziming:} #1}}}%
  \endgroup
}

\let\chapter\section

%==============================================================================
\pagestyle{plain}
%==============================================================================

\title{Protected Automotive Remote Entry Device (PARED) \\ System Design}
\author{MITRE eCTF 2023\\Team \textbf{Cacti}\\ University at Buffalo}
\date{}



\begin{document}
%%
%=============================================================================
\normalsize


\maketitle
%\date{}

\renewcommand{\thepage}{System Design, Team Cacti, University at Buffalo--\arabic{page}}
\setcounter{page}{1} \normalsize
%
%\renewcommand{\baselinestretch}{1.2}
%\normalsize
%\vspace{0.1in}
%\centerline{\textbf{\Large }}
%\renewcommand{\baselinestretch}{1.0}
%\normalsize

\newcommand{\flagRollback}{\textsf{Rollback}\xspace}

\section{Introduction}

This section presents the entities and communication channels in the system.

\subsection{Entities}

The following summarizes the entities in the system.

\section{Security Requirements}

This section defines the security requirements of our design.

\subsection{SR1}
\textbf{A car should only unlock and start when the user has an authentic fob that is paired with the car.}

In the reference design, ...

\paragraph{How we address it:} ...

\subsection{SR2}
\textbf{Revoking an attacker's physical access to a fob should also revoke their ability to unlock the associated car.}

In the reference design, ...

\paragraph{How we address it:} ...

\subsection{SR3}
\textbf{Observing the communications between a fob and a car while unlocking should not allow an attacker to unlock the car in the future.}

In the reference design, ...

\paragraph{How we address it:} ...

\subsection{SR4}
\textbf{Having an unpaired fob should not allow an attacker to unlock a car without a corresponding paired fob and pairing PIN.}

In the reference design, ...

\paragraph{How we address it:} ...

\subsection{SR5}
\textbf{A car owner should not be able to add new features to a fob that did not get packaged by the manufacturer.}

In the reference design, ...

\paragraph{How we address it:} ...

\subsection{SR6}
\textbf{Access to a feature packaged for one car should not allow an attacker to enable the same feature on another car.}

In the reference design, ...

\paragraph{How we address it:} ...



\section{Security Implementations}



\subsection{Build PARED System}

\subsubsection{Build Environment}

This step will build the docker image from the Dockerfile. We will list all the additional packages we used besides those are in the reference design.

\subsubsection{Build Tools}

\textit{The resulting host tools will be given to the other teams in the Attack Phase.}

Our host tools are written in Python, thus this step will simply copy all the tools in to the docker container.

\subsubsection{Build Deployment}

\textit{Attackers will NEVER be given access to the Host Secrets generated in this step.}

In our design, the Host Secrets contain two parts: the Global Secret and the Car Secrets, and they will be saved in the "secrets" docker volume.
This step will generate the Global Secret, which is a deployment-wide secret shared across cars and/or fobs.

We generate the Global Secret randomly, to prevent the attacker from retrieving it through reverse engineering.

\subsubsection{Build Car, Paired Fob, and Unpaired Fob}

\textit{The plaintext firmware and EEPROM files produced in these steps are not given to attackers except for the car and paired fob of Car 0 that will not contain any flags.}

\textit{These build steps may read and modify the Host Secrets.}

To build the car binaries, a car ID will be supplied as a flag to the building tools. We will:
\begin{itemize}
	\item Check whether the car secret that corresponding to this car ID exists in the "secrets" docker volume.
	\item If it not exists, we randomly generate one, and save it into the "secrets" docker volume.
	\item This car secret will get loaded into the car EEPROM data. The car ID will be saved in the car firmware.
\end{itemize}

To build the paired fob binaries, a car ID (corresponding to this fob) and a fob pairing PIN will be supplied as a flag to the building tools. We will:
\begin{itemize}
	\item Retrieve the car secret which corresponding to this car ID from the "secrets" docker volume.
	\item Retrieve the global secret from the "secrets" docker volume.
	\item Save the car secret and global secret into the paired fob EEPROM data. Save the car ID and fob pairing PIN in the paired fob firmware.
\end{itemize}

To build the unpaired fob binaries, we will:
\begin{itemize}
	\item Retrieve the global secret from the "secrets" docker volume.
	\item Save the global secret into the paired fob EEPROM data.
\end{itemize}

\subsection{Load Devices}

\textit{Teams will not be able to modify any part of this step.}

After building the system, the firmware and EEPROM contents are loaded onto the microcontrollers by the provided tools.

\subsection{Host Tools}

\subsubsection{Package Feature}

\textit{Attackers will be given access to the packaged feature produced in this step in many scenarios.}




\end{document}
%==============================================================================
